\documentclass{scrartcl}
\usepackage{graphicx,amsmath,latexsym}
\usepackage{color,psfrag,boxedminipage,amssymb}
%\usepackage[latin1]{inputenc}
%\usepackage[swedish]{babel}
%\usepackage{draftcopy}
\usepackage{tikz}
\usepackage{pgf}
\usetikzlibrary{positioning,arrows}

\newcommand{\st}{\ensuremath{\boldsymbol{s}_k(\boldsymbol{\theta})}}
\newcommand{\St}{{\mathbf S}\ensuremath{(\boldsymbol{\theta})}}
\newcommand{\phibf}{\ensuremath{\boldsymbol{\phi}}}
\newcommand{\varphibf}{\ensuremath{\boldsymbol{\varphi}}}
\def \phibfh{\ensuremath{\hat{\boldsymbol{\phi}}}}
\renewcommand{\sp}{\ensuremath{\boldsymbol{s}(\boldsymbol{\phi})}}
\newcommand{\hp}{\ensuremath{\boldsymbol{h}(\boldsymbol{\phi})}}
\newcommand{\thk}{\ensuremath{\boldsymbol{\theta}_k}}
\renewcommand{\th}{\ensuremath{\boldsymbol{\theta}}}
\newcommand{\alp}{\ensuremath{\boldsymbol{\alpha}}}
\def \thh{\ensuremath{\hat{\boldsymbol{\theta}}}}
\def \epsbf{\ensuremath{\boldsymbol{\epsilon}}}
\def \tht{\tilde{\th}}
\newcommand{\mubf}{\ensuremath{\boldsymbol{\mu}}}

\newcommand{\ctft}{\buildrel{{\cal F}}\over{\longleftrightarrow}}
\newcommand{\defin}{\buildrel{\triangle}\over{=}}

\def \Em{{\mathbb{E}}}
\def \hh{{\mathcal{H}}}
\def \cw{{(\lceil 2^Q || w ||\rceil)}}



\def \abf{{\mathbf a}}
\def \Abf{{\mathbf A}}
\def \bbf{{\mathbf b}}
\def \Bbf{{\mathbf B}}
\def \cbf{{\mathbf C}}
\def \Cbf{{\mathbf C}}
\def \dbf{{\mathbf d}}
\def \Dbf{{\mathbf D}}
\def \ebf{{\mathbf e}}
\def \Ebf{{\mathbf E}}
\def \fbf{{\mathbf f}}
\def \Fbf{{\mathbf F}}
\def \gbf{{\mathbf g}}
\def \Gbf{{\mathbf G}}
\def \hbf{{\mathbf h}}
\def \Hbf{{\mathbf H}}
\def \ibf{{\mathbf i}}
\def \Ibf{{\mathbf I}}
\def \jbf{{\mathbf j}}
\def \Jbf{{\mathbf J}}
\def \kbf{{\mathbf k}}
\def \Kbf{{\mathbf K}}
\def \lbf{{\mathbf l}}
\def \Lbf{{\mathbf L}}
\def \mbf{{\mathbf m}}
\def \Mbf{{\mathbf M}}
\def \nbf{{\mathbf n}}
\def \Nbf{{\mathbf N}}
\def \obf{{\mathbf o}}
\def \Obf{{\mathbf O}}
\def \pbf{{\mathbf p}}
\def \Pbf{{\mathbf P}}
\def \qbf{{\mathbf q}}
\def \Qbf{{\mathbf Q}}
\def \rbf{{\mathbf r}}
\def \Rbf{{\mathbf R}}
\def \sbf{{\mathbf s}}
\def \Sbf{{\mathbf S}}
\def \tbf{{\mathbf t}}
\def \Tbf{{\mathbf T}}
\def \ubf{{\mathbf u}}
\def \Ubf{{\mathbf U}}
\def \vbf{{\mathbf v}}
\def \Vbf{{\mathbf V}}
\def \wbf{{\mathbf w}}
\def \Wbf{{\mathbf W}}
\def \xbf{{\mathbf x}}
\def \Xbf{{\mathbf X}}
\def \ybf{{\mathbf y}}
\def \Ybf{{\mathbf Y}}
\def \zbf{{\mathbf z}}
\def \Zbf{{\mathbf Z}}
\def \0bf{{\mathbf 0}}

\def \Emean{\mathbb{E}}

\def \Sibf{{\mathbf \Sigma}}
\def \xbbf{\mathbf{\bar{x}}}
\def \etr{\mbox{etr}}
\def \tr{\mbox{tr}}
\def \Tr{\mbox{Tr}}
\def \Cov{\mbox{Cov}}
\def \cost{\mbox{cost}}
\def \diag{\mbox{diag}}
\def \Lambf{{\mathbf{\Lambda}}}
\def \Gambf{{\mathbf{\Gamma}}}
\def \Sigbf{{\mathbf \Sigma}}
\newcommand{\rhobf}{\ensuremath{\boldsymbol{\rho}}}
\newcommand{\lambf}{\ensuremath{\boldsymbol{\lambda}}}
\newcommand{\nubf}{\ensuremath{\boldsymbol{\nu}}}

\newcounter{examplenr}[section]
\renewcommand{\theexamplenr}{\arabic{examplenr}}%{\thesection.\arabic{examplenr}}
\newenvironment{example}[1]{\vskip \baselineskip
\refstepcounter{examplenr}\noindent{{\bf
Example~\theexamplenr}\hskip .5em #1\\} }{\hrulefill $\Box$
 \vskip\baselineskip}



\newcounter{theoremnr}[section]
%\renewcommand{\thetheoremnr}{\thesection.\arabic{theoremnr}}
\renewcommand{\thetheoremnr}{\arabic{theoremnr}}
\newtheorem{A}[theoremnr]{Theorem}
\newcounter{theoremProofnr}[section]
%\renewcommand{\thetheoremProofnr}{\thesection.\arabic{theoremProofnr}}
\renewcommand{\thetheoremProofnr}{\arabic{theoremProofnr}}
\newtheorem{B}[theoremProofnr]{Proof of Theorem}
\newcounter{corollarynr}[section]
\renewcommand{\thecorollarynr}{\thesection.\arabic{corollarynr}}
\newtheorem{C}[corollarynr]{Corollary}
\newcounter{Lemmanr}[section]
\renewcommand{\theLemmanr}{\thesection.\arabic{Lemmanr}}
\newtheorem{D}[Lemmanr]{Lemma}
\newcounter{LemmaProofnr}[section]
\renewcommand{\theLemmaProofnr}{\thesection.\arabic{LemmaProofnr}}
\newtheorem{E}[LemmaProofnr]{Proof of Lemma}

\newcounter{propositionnr}[section]
\renewcommand{\thepropositionnr}{\arabic{propositionnr}}
\newtheorem{F}[propositionnr]{Proposition}
\newcounter{propositionProofnr}[section]
\renewcommand{\thepropositionProofnr}{\arabic{propositionProofnr}}
\newtheorem{G}[propositionProofnr]{Proof of proposition}

\newenvironment{working}{\color{blue}\sffamily\em}{}
\newenvironment{forslag}{\color{red}\sffamily\em}{}

\newcounter{definnr}[section]
\renewcommand{\thedefinnr}{\arabic{definnr}}
\newtheorem{J}[definnr]{Definition}

\usepackage{amsmath}
\usepackage{geometry}
\usepackage{caption}
\usepackage{lipsum}
\usepackage{hhline}
\geometry{a4paper}
\usepackage[backend=biber,style=ieee]{biblatex}
\bibliography{ref.bib}
\usepackage{comment}
\usepackage{multirow,array,units}

\usepackage{fancyhdr}

\pagestyle{fancy}
\fancyhf{}
\rhead{bvc981}
\lhead{Dhruv Chauhan}

\newenvironment{redmatrix}
  {\left(\array{@{}rrr|ccc@{}}}
  {\endarray\right)}
\newenvironment{ropmatrix}
  {\array{@{}c@{}}}
  {\endarray}
\newcommand\opone[2]{\xrightarrow{(#1)\times r_#2}}
\newcommand\optwo[3]{\xrightarrow{r_#1{}+{} #2r_#3}}
\begin{document}
\title{Machine Learning Final Exam}
\subtitle{Department of Computer Science, University of Copenhagen}
\author{Dhruv Chauhan}
\maketitle

\section{In a galaxy far, far away}
\subsection{}
The variance of the red-shifts in the spectroscopic training data was calculated to be:
\[ 0.0106 \]
(where from now on, unless specified, values are shown to 3 significant places). \\

The MSE on the test SDSS predictions was calculated to be:
\[ 0.000812 \]
This shows that the predictions were quite accurate.

\subsection{}

The linear regression was done in Python, using the \texttt{sklearn} linear regression package. This performs an ordinary least squares linear regression. The error function is a Mean Squared Error. \\

The parameters of the model were (taken from the announcement):
\begin{center}
[  0.0185134,0.0479647,-0.0210943,-0.0274002, \\
-0.0226798,0.0064449,0.0151842,0.0120738, \\
0.0103486,0.00599684,-0.0294513,0.069059, \\
0.00630583,-0.00472042,-0.00873932,0.00311043, \\
0.0017252,0.00435176]
\end{center}

*** CHECK AND MAYBE PLUG THESE INTO THE MODEL ***

with bias term:

\[ -0.801881 \]
The error on the training data was calculated to be $0.00187$, and on the test data was $0.00187$ also. The errors noramlised by the variance, $\sigma^2_{red}$ were equal to $0.176$ for both the test and the training data. \\

% FINISH
This normalised error falling below one signifies that...

\subsection{}

For the non-linear regression, I chose to apply the K-nearest neighbours (KNN) algorithm. I chose this method for its simplicity (following Occam's razor), and therefore its intuitive understanding. The simplicity of the algorithm is also reflected in the single hyperparameter, $k$ (if you consider a fixed distance metric), which means that there is less computation in tuning the hyperparameter. \\

I utilised the \texttt{neighbours} library from the \texttt{sklearn} package. \\

The KNN algorithm uses a distance metric to calculate the distance between a (set of) training point(s) and the other points. I used the Euclidian distance, given by $ || \xbf - \xbf' || $, or $ \sqrt {\xbf^T \xbf' } $. The algorithm works by calculating the distances from a test point to the other points, and then finding the nearest K points to that point. In a regression task, the test point is assigned the value of the mean of the nearest K neighbours. \\

My method involved using model selection methods such as cross-validation and grid search. Since this is model selection, cross validation was needed as we only use the training data during model selection. This gave us a better way to prevent overfitting of the training data. I used the \texttt{GridSearchCV} package from \texttt{sklearn}. The range of possible $k$ values was given as the odd numbers between 1 and 29. This performed a 5-fold cross validation on each of the possible values of $k$, averaging out the resulting error (i.e. splitting the data into 5 equal chunks, using 4 as the training set, and 1 as the validation set, and then cycling through all possible 5 validation sets). The error function used in the algorithm was the mean squared error.\\

This method resulted in the optimum hyperparameter as $k = 7$, with a MSE of $0.00118$ on the test data, and a MSE of $0.000870$. \\

Clearly, the KNN Regressor worked better on the training data, which is to be expected due to the model's simplicity in using the $k$ training points' average to return the regression results - therefore the training points are bound to have a low MSE. In comparison, the training data in the linear regressor performed the same as the test data. The difference in this would be due to the nature of a linear regressor, which would 'average out' the regression line over the points, thus leading to a small MSE on both the training and test data. On the test data, the KNN Regressor did perform a bit better than the linear regressor - perhaps due to the non-linear nature of KNN capturing the underlying nature of the data slightly more accurately. Overall, I believe the KNN method worked well as we had a relatively low variance on the data - KNN can be skewed by big outliers in the data. The cross-validation helped to prevent overfitting on the data, which may have also caused the error on the test data to drop in comparison to the linear regressor.

\newpage
\section{Weed}

\subsection{}
The logistic regression was done in python, using the \texttt{LogisticRegression} package from \texttt{sklearn}. This implementation uses the logistic function. We are trying to classify testing points according to binary labels (0 - weeds, 1 - crops). The algorithm tries to minimise the following function:

\[ \min_{w, c} \frac{1} {2} w^T w + C \sum^n_{i=1} \log e ^{(-y_i(X^t_i \; w + c)) +1)} \]

It also uses a coordinate descent algorithm, which is a derivative-free optimization algorithm that performs a line search in one coordinate direction for the current point in each iteration. (Wikipedia, The Free Encyclopedia, 2016)\\

The parameters produced by the model were:

\begin{center}
[  -0.0391283 ,  0.01549887,  0.00295803,  0.00033714, \\
   -0.00039954,  0.00348062, -0.00715483,  0.00473467, \\
   -0.02685346, -0.05592747, -0.04317431,  0.00579407, \\
  -0.00998736
]
\end{center}

with bias term:

\[ -1.47771341e-05 \]

The zero-one loss on the training data was: $0.0200$, and on the test data: $0.0348$.

\subsection{}
I used the \texttt{sklearn} python package, specfically the \texttt{svm.SVC} and \texttt{GridSearchCV} packages - in addition to \texttt{numpy}. Their model selection allows cross validation and grid-searching across values. It does this by splitting the training data into the training and validation sets across 5 folds and then running with the specified grid combination, for each option, finally returning the option with the lowest classification error. The error was calculated as the zero-one loss.\\

I used a kernel of the form:

\[ k(\xbf, \zbf) = exp \big(-\gamma {|| \xbf - \zbf ||}^2\big) \] \\

\newpage
Using the formula given, I calculated:

\begin{align*}
    \sigma_{\text{Jaakkola}} &= 609 \\
    \gamma_{\text{Jaakkola}} &= 1.35e-06
\end{align*}

From that, the grid search looked over values of C and $\gamma$ as given in the instructions, with $b = 10$. \\

From the grid search, the optimum hyperparameters were found to be:

\begin{align*}
    C &= 1000 \\
    \gamma &= 1.35e-8
\end{align*}

The accuracy for the training and test sets is shown below:

\begin{align*}
    & \text{accuracy}_{\text{training}}   &= 0.978 \\
    & \text{accuracy}_{\text{test}}       &= 0.969 \\
\end{align*}

From the accuracy results above, on both the training and test sets, the SVM performed very well overall. This demonstrates the effectiveness of SVMs, especially in comparison to logistic regression.

% Obligatory citation, and I decided to point to  your text book~\cite{abu2012learning}. The references sit in a separate file, ref.bib.
%
% \printbibliography
\end{document}
