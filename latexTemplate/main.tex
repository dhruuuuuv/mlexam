\documentclass{scrartcl}
\usepackage{graphicx,amsmath,latexsym}
\usepackage{color,psfrag,boxedminipage,amssymb}
%\usepackage[latin1]{inputenc}
%\usepackage[swedish]{babel}
%\usepackage{draftcopy}
\usepackage{tikz}
\usepackage{pgf}
\usetikzlibrary{positioning,arrows}

\newcommand{\st}{\ensuremath{\boldsymbol{s}_k(\boldsymbol{\theta})}}
\newcommand{\St}{{\mathbf S}\ensuremath{(\boldsymbol{\theta})}}
\newcommand{\phibf}{\ensuremath{\boldsymbol{\phi}}}
\newcommand{\varphibf}{\ensuremath{\boldsymbol{\varphi}}}
\def \phibfh{\ensuremath{\hat{\boldsymbol{\phi}}}}
\renewcommand{\sp}{\ensuremath{\boldsymbol{s}(\boldsymbol{\phi})}}
\newcommand{\hp}{\ensuremath{\boldsymbol{h}(\boldsymbol{\phi})}}
\newcommand{\thk}{\ensuremath{\boldsymbol{\theta}_k}}
\renewcommand{\th}{\ensuremath{\boldsymbol{\theta}}}
\newcommand{\alp}{\ensuremath{\boldsymbol{\alpha}}}
\def \thh{\ensuremath{\hat{\boldsymbol{\theta}}}}
\def \epsbf{\ensuremath{\boldsymbol{\epsilon}}}
\def \tht{\tilde{\th}}
\newcommand{\mubf}{\ensuremath{\boldsymbol{\mu}}}

\newcommand{\ctft}{\buildrel{{\cal F}}\over{\longleftrightarrow}}
\newcommand{\defin}{\buildrel{\triangle}\over{=}}

\def \Em{{\mathbb{E}}}
\def \hh{{\mathcal{H}}}
\def \cw{{(\lceil 2^Q || w ||\rceil)}}



\def \abf{{\mathbf a}}
\def \Abf{{\mathbf A}}
\def \bbf{{\mathbf b}}
\def \Bbf{{\mathbf B}}
\def \cbf{{\mathbf C}}
\def \Cbf{{\mathbf C}}
\def \dbf{{\mathbf d}}
\def \Dbf{{\mathbf D}}
\def \ebf{{\mathbf e}}
\def \Ebf{{\mathbf E}}
\def \fbf{{\mathbf f}}
\def \Fbf{{\mathbf F}}
\def \gbf{{\mathbf g}}
\def \Gbf{{\mathbf G}}
\def \hbf{{\mathbf h}}
\def \Hbf{{\mathbf H}}
\def \ibf{{\mathbf i}}
\def \Ibf{{\mathbf I}}
\def \jbf{{\mathbf j}}
\def \Jbf{{\mathbf J}}
\def \kbf{{\mathbf k}}
\def \Kbf{{\mathbf K}}
\def \lbf{{\mathbf l}}
\def \Lbf{{\mathbf L}}
\def \mbf{{\mathbf m}}
\def \Mbf{{\mathbf M}}
\def \nbf{{\mathbf n}}
\def \Nbf{{\mathbf N}}
\def \obf{{\mathbf o}}
\def \Obf{{\mathbf O}}
\def \pbf{{\mathbf p}}
\def \Pbf{{\mathbf P}}
\def \qbf{{\mathbf q}}
\def \Qbf{{\mathbf Q}}
\def \rbf{{\mathbf r}}
\def \Rbf{{\mathbf R}}
\def \sbf{{\mathbf s}}
\def \Sbf{{\mathbf S}}
\def \tbf{{\mathbf t}}
\def \Tbf{{\mathbf T}}
\def \ubf{{\mathbf u}}
\def \Ubf{{\mathbf U}}
\def \vbf{{\mathbf v}}
\def \Vbf{{\mathbf V}}
\def \wbf{{\mathbf w}}
\def \Wbf{{\mathbf W}}
\def \xbf{{\mathbf x}}
\def \Xbf{{\mathbf X}}
\def \ybf{{\mathbf y}}
\def \Ybf{{\mathbf Y}}
\def \zbf{{\mathbf z}}
\def \Zbf{{\mathbf Z}}
\def \0bf{{\mathbf 0}}

\def \Emean{\mathbb{E}}

\def \Sibf{{\mathbf \Sigma}}
\def \xbbf{\mathbf{\bar{x}}}
\def \etr{\mbox{etr}}
\def \tr{\mbox{tr}}
\def \Tr{\mbox{Tr}}
\def \Cov{\mbox{Cov}}
\def \cost{\mbox{cost}}
\def \diag{\mbox{diag}}
\def \Lambf{{\mathbf{\Lambda}}}
\def \Gambf{{\mathbf{\Gamma}}}
\def \Sigbf{{\mathbf \Sigma}}
\newcommand{\rhobf}{\ensuremath{\boldsymbol{\rho}}}
\newcommand{\lambf}{\ensuremath{\boldsymbol{\lambda}}}
\newcommand{\nubf}{\ensuremath{\boldsymbol{\nu}}}

\newcounter{examplenr}[section]
\renewcommand{\theexamplenr}{\arabic{examplenr}}%{\thesection.\arabic{examplenr}}
\newenvironment{example}[1]{\vskip \baselineskip
\refstepcounter{examplenr}\noindent{{\bf
Example~\theexamplenr}\hskip .5em #1\\} }{\hrulefill $\Box$
 \vskip\baselineskip}



\newcounter{theoremnr}[section]
%\renewcommand{\thetheoremnr}{\thesection.\arabic{theoremnr}}
\renewcommand{\thetheoremnr}{\arabic{theoremnr}}
\newtheorem{A}[theoremnr]{Theorem}
\newcounter{theoremProofnr}[section]
%\renewcommand{\thetheoremProofnr}{\thesection.\arabic{theoremProofnr}}
\renewcommand{\thetheoremProofnr}{\arabic{theoremProofnr}}
\newtheorem{B}[theoremProofnr]{Proof of Theorem}
\newcounter{corollarynr}[section]
\renewcommand{\thecorollarynr}{\thesection.\arabic{corollarynr}}
\newtheorem{C}[corollarynr]{Corollary}
\newcounter{Lemmanr}[section]
\renewcommand{\theLemmanr}{\thesection.\arabic{Lemmanr}}
\newtheorem{D}[Lemmanr]{Lemma}
\newcounter{LemmaProofnr}[section]
\renewcommand{\theLemmaProofnr}{\thesection.\arabic{LemmaProofnr}}
\newtheorem{E}[LemmaProofnr]{Proof of Lemma}

\newcounter{propositionnr}[section]
\renewcommand{\thepropositionnr}{\arabic{propositionnr}}
\newtheorem{F}[propositionnr]{Proposition}
\newcounter{propositionProofnr}[section]
\renewcommand{\thepropositionProofnr}{\arabic{propositionProofnr}}
\newtheorem{G}[propositionProofnr]{Proof of proposition}

\newenvironment{working}{\color{blue}\sffamily\em}{}
\newenvironment{forslag}{\color{red}\sffamily\em}{}

\newcounter{definnr}[section]
\renewcommand{\thedefinnr}{\arabic{definnr}}
\newtheorem{J}[definnr]{Definition}

\usepackage{amsmath}
\usepackage{geometry}
\usepackage{caption}
\usepackage{lipsum}
\usepackage{hhline}
\geometry{a4paper}
\usepackage[backend=biber,style=ieee]{biblatex}
\bibliography{ref.bib}
\usepackage{comment}
\usepackage{multirow,array,units}

\usepackage{fancyhdr}

\pagestyle{fancy}
\fancyhf{}
\rhead{bvc981}
\lhead{Dhruv Chauhan}

\newenvironment{redmatrix}
  {\left(\array{@{}rrr|ccc@{}}}
  {\endarray\right)}
\newenvironment{ropmatrix}
  {\array{@{}c@{}}}
  {\endarray}
\newcommand\opone[2]{\xrightarrow{(#1)\times r_#2}}
\newcommand\optwo[3]{\xrightarrow{r_#1{}+{} #2r_#3}}
\begin{document}
\title{Machine Learning Final Exam}
\subtitle{Department of Computer Science, University of Copenhagen}
\author{Dhruv Chauhan}
\maketitle

\section{In a galaxy far, far away}
\subsection{}
The variance of the red-shifts in the spectroscopic training data was calculated to be:

\[ 0.0106 \]

(where from now on, unless specified, values are shown to 3 significant places). \\

The MSE on the test SDSS predictions was calculated to be:

\[ 0.000812 \]

\subsection{}

The linear regression was done in Python, using the \texttt{sklearn} linear regression package. This performs an ordinary least squares linear regression. The error function is a Mean Squared Error. \\

The parameters of the model were calculated to be:

\begin{center}
[  -2.82898070e+11,   6.79638352e+11, -7.30280682e+11,  3.84379194e+11, \\
   -5.08387940e+10,  -2.44829466e+11,  6.20014394e+10,  4.45567121e+11, \\
   -3.89498568e+11,   1.26759474e+11,  2.82898070e+11, -3.96740282e+11, \\
    3.33540400e+11,  -5.08387940e+10,  2.44829466e+11,  1.82828027e+11, \\
   -2.62739094e+11,   1.26759474e+11]
\end{center}

The error on the training data was calculated to be $0.00187$, and on the test data was $0.00187$ also. The errors noramlised by the variance, $\sigma^2_{red}$ were equal to $0.176$ for both the test and the training data. \\

% FINISH
This normalised error falling below one signifies that...

\subsection{}

For the non-linear regression, I chose to apply the K-nearest neighbours (KNN) algorithm. I chose this method for its simplicity (following Occam's razor), and therefore its ease of understanding. The simplicity of the algorithm is also reflected in the single hyperparameter, $k$, which means that there is less computation in tuning the hyperparameter. \\

I utilised the \texttt{neighbours} library from the \texttt{sklearn} package.

The KNN algorithm uses a distance metric to calculate the distance between a (set of) training point(s). I used the Euclidian distance, given by $ || \xbf - \xbf' || $, or $ \sqrt {\xbf^T \xbf' } $.

My method involved doing the following:
\begin{enumerate}
    \item Given a certain K,
\end{enumerate}

% Obligatory citation, and I decided to point to  your text book~\cite{abu2012learning}. The references sit in a separate file, ref.bib.
%
% \printbibliography
\end{document}
